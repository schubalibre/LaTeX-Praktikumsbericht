% !TEX root = ../praktikumsbericht.tex
\chapter{Tätigkeitsbereiche und Aufgaben}\label{chap:aufgaben}

Wie bereits erwähnt, hat sich schon zu meiner Zeit als Werkstudentin bei Wall herausgestellt, dass es einige Schnittstellentätigkeiten zwischen der IT-Abteilung und dem Marketing gibt, bei denen mein Team aus dem digitalen Marketing Unterstützung gebrauchen könnte. Vor dem Hintergrund, dass diese Bereiche durch den digitalen Wandel zunehmend miteinander verschmelzen, werden die Aufgaben oft komplexer und die Klärung des Zuständigkeitsbereichs gestaltet sich schwieriger. An diesem Punkt setzt mein Praktikum an: ich sollte Arbeitsaufträge erhalten, die sowohl im Digitalen Marketing, als auch in der IT Anwendung finden, denn dort ist Zusammenarbeit Pflicht. \\ Da sich meine Aufgabenbereiche als sehr vielfältig herausgestellt haben und es keine große Kernaufgabe gab, an der ich für drei Monate arbeiten sollte, geben die folgenden Abschnitte einen Einblick in die Arbeitsbereiche, mit denen ich mich am meisten beschäftigte. 

\section{Einarbeitung}\label{sec:einarbeitung}

Zu Beginn meines Praktikums wurde ich zunächst allen Personen aus dem Marketing und der IT-Abteilung vorgestellt, mit denen ich künftig zusammenarbeiten würde. Insgesamt empfand ich es als sehr aufmerksam und respektvoll, wie man mich an die verschiedenen Themenbereiche heranführte. Für die Einarbeitung nahmen sich meine Betreuer stets viel Zeit und legten Wert darauf, mich mit den jeweiligen Tools intensiv vertraut zu machen. Neben einer eigenen E-Mail-Adresse erhielt ich auch eigene Accounts und Zugänge für den alten und neuen Webseiten-Bereich sowie für das Intranet. Zusätzlich erhielt ich einige hausinternen, aber auch externen Infomaterialien zum Unternehmen und dem dazugehörenden Portfolio. \\ Da ich mit verschiedenen Tools und Content-Management-Systemen arbeiten würde, hat die Einarbeitungsphase möglicherweise mehr Zeit eingenommen, als vorher angenommen. Zur Einarbeitung erhielt ich meistens kleinere, nicht zeitgebundene Aufgaben; z.B. das Austauchen von Kontakten und Bildern auf der aktuellen Webseite oder die Aktualisierung von Daten im Intranet. Nachdem mir die grundlegende Funktionsweise der Systeme nähergebracht wurde, konnte ich viele Aufgaben nach dem \textit{Learning by Doing}-Prinzip bewerkstelligen, was mich stets motiviert und einen großen Lerneffekt zur Folge hatte. \\ Obwohl ich zu Beginn meines Praktikums schon seit drei Monaten im Unternehmen arbeitete, erwies sich die Einarbeitungsphase als angemessen, da viele Aufgabenfelder für mich noch neu waren. Trotzdem waren die drei vorangegangenen Monate als Werkstudent hilfreich, um mich einzugewöhnen und die wichtigsten Strukturen und Konzepte des Unternehmens zu verstehen. 

\section{Website Relaunch}\label{sec:webrelaunch}

Den wohl größten und intensivsten Teil meines Praktikums nimmt der Relaunch der Website von Wall ein. Das neue Website-Konzept und Redesign von wall.de beschäftigte mich während der gesamten Praktikumszeit. Da mein Team fast ausschließlich am Website-Projekt arbeitete, dienten viele meiner anderen Aufgabenbereiche dazu, das Team dahingehend zu entlasten. Dadurch bot sich für mich die Möglichkeit, sowohl an einem großen Vorhaben mitwirken zu können, als auch andere Teilbereiche abzudecken und kennenzulernen. \\ Zu Beginn meines Praktikums arbeitete mein Team bereits seit über einem Jahr an der neuen Website und ich stieg somit zu einem Zeitpunkt in das Projekt ein, an dem große Teile des Konzepts bereits feststanden und auch ein Teil des Contents schon eingebunden wurde. \\ Meine Aufgabe war es nun, bei der Gestaltung und der Konzeptionierung der noch offenen Seiten mitzuhelfen: Von der Idee über ein Konzept und den Entwurf bis hin zur Umsetzung mit Blick auf das Front-End wurde ich in alle Prozesse einbezogen. \\ Die neue Website wurde mit dem Content-Management-System und Framework Drupal entwickelt. Die Entscheidung, mit diesem CMS zu arbeiten, wurde nicht innerhalb unserer Abteilung getroffen, sondern entspricht den Rahmenbedingungen von JCDecaux weltweit. Als Teil dieses internationalen Unternehmens war es das Ziel, sich am Look \& Feel der offiziellen Website von jcdecaux, \textit{www.jcdecaux.com} zu orientieren und daher mit demselben Framework zu arbeiten. 


\section{Event Management}\label{sec:eventmanagement}
\section{Sonstige Aufgaben}\label{sec:sonstiges}
\section{Praktikum und Studium}\label{sec:praktikumstudium}
\subsection{Wissensanwendung im Praktikum}\label{sec:wissensanwendung}
\subsection{Bedeutung für die Zukunft}\label{sec:zukunft}

\newpage