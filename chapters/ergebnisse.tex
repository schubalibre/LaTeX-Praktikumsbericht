% !TEX root = ../maturaarbeit.tex
\chapter{Ergebnisse}\label{chap:ergebnisse}
Die Ergebnisse einer anderen Arbeit sind in \cite{Boney96} dokumentiert.
\LaTeX-Dateien können auch einfach per \verb+\input{pfad/zum/dokument.tex}+ eingebunden werden, so wie dies mit \fref{tab:tabelle2} geschehen ist.
% !TEX root = ../maturaarbeit.tex
\begin{table}[th]
  \centering
  \footnotesize
  \caption{Anwendung von Linien}
  \label{tab:tabelle2}
  \begin{tabular}{@{}l*{4}{d{4.0}}@{}}
    \toprule
      Monat & 1965 & 1966 & 1967 & 1968 \\
    \cmidrule(r){1-1}\cmidrule(lr){2-2}\cmidrule(lr){3-3}\cmidrule(lr){4-4}%
      \cmidrule(l){5-5}
      November  & 2500 & 2800 & 4700 & 3200 \\
      Dezember  & 2300 & 2000 & 3600 & 2700 \\
    \bottomrule
  \end{tabular}
\end{table}
Einige andere Ergebnisse finden sich in \fref{tab:tabelle1}.\\
Wenn man \LaTeX\ nicht zu fest dreinredet, macht es was es soll.

\begin{table}
  \centering
  \caption[CUDA Messdaten]{Ein paar CUDA Messdaten}
  \label{tab:tabelle1}
  \footnotesize
  \begin{tabular}{@{}ld{3.0}*{2}{d{4.2}}d{2.2}@{}}
    \toprule
    \multicolumn{0}{@{}l}{Method} & \multicolumn{1}{c}{\# Calls} & \multicolumn{1}{c}{GPU time [\textmu s]} & \multicolumn{1}{c}{CPU time [\textmu s]} & \multicolumn{1}{c@{}}{GPU time [\%]}\\
    \midrule
    sumvecel\_kernel&386&3095.55&3613.56&19.84\\
    matvec\_kernel&193&2587.87&2848.87&16.58\\
    mattransvec\_kernel&193&2566.69&2792.69&16.45\\
    scalvec\_kernel&386&2322.11&2778.11&14.88\\
    squarevecel\_kernel&386&2098.62&2547.62&13.45\\
    sasum\_gld\_main&193&2044.48&2299.48&13.10\\
    memcpyDtoH&202&831.52&3632.52&5.32\\
    memcpyHtoD&7&28.54&36.54&0.18\\
    sger\_main\_hw&3&27.04&30.04&0.17\\
    \bottomrule
  \end{tabular}
\end{table}
\newpage