% !TEX root = ../praktikumsbericht.tex
\chapter{Praktikumsbetrieb}\label{chap:praktikumsbetrieb}
\section{Wall GmbH}\label{sec:wallgmbh}

„Die Wall GmbH ist ein deutsches Unternehmen mit Sitz in Berlin, das sich auf die Entwicklung und Produktion von Stadtmöbeln und die Vermarktung hinterleuchteter Werbeflächen spezialisiert hat.“\footnote{\label{foot:1}Wikipedia: „Wall GmbH“, unter: http://de.wikipedia.org/wiki/Wall\_GmbH (abgerufen am 03.12.2017)} Seit 2009 ist die Wall GmbH Teil der internationalen JCDecaux-Gruppe, der Nummer 1 der Außenwerbung weltweit. JCDecaux erzielte im Jahr 2015 einen Umsatz von 3.207 Millionen Euro in mehr als 60 Ländern mit 12.850 Mitarbeitern, von denen mehr als 1.000 Mitarbeiter zur Wall GmbH zählen.\footnote{\label{foot:2}Wall GmbH: „Fakten“, unter: http://www.wall.de/de/company/facts (abgerufen am 03.12.2017)} Die Wall GmbH bietet Städten im Rahmen von Stadtverträgen individuell konzipierte Stadtmöbel an, die sie kostenfrei installiert, reinigt und wartet. Das Unternehmen refinanziert die kostenfreien Produkte und Dienstleistungen über die Vermarktung der in die Stadtmöbel integrierten Werbeflächen. Darüber hinaus werden die Städte an den Einnahmen der Außenwerbung beteiligt.\footnote{\label{foot:3}Berliner Morgenpost: „Berliner Wall AG will sich komplett digital aufstellen“, unter: https://www.morgenpost.de/berlin/article132236230/Berliner-Wall-AG-will-sich-komplett-digital-aufstellen.html (abgerufen am 06.01.2018)} \\ Das Unternehmen möchte mit seinen hochwertigen Produkten die Informations- und Lebensqualität für die Bürger und Besucher der Städte im öffentlichen Raum verbessern. Das Angebot erstreckt sich von leistungsstarken Werbenetzen über U-Bahnhöfe bis hin zu Transportmitteln. Insgesamt vermarktet die Wall GmbH in Deutschland mehr als 76.000 Werbeflächen (Stand: 31. Dezember 2016).\footnote{\label{foot:4}Wall GmbH: „Fakten“, unter: http://www.wall.de/de/company/facts (abgerufen am 06.01.2018)} 

\subsection{Unternehmenskonzept}\label{sec:unternehmenskonzept}
Wall setzt, im Gegensatz zu seinem großen Konkurrenten \textit{Ströer}, auf das Konzept „Alles aus einer Hand“. Die Stadtmöbel entstehen in einem eigenen Forschungs- und Entwicklungszentrum in Velten bei Berlin. In dem eigenen Produktionswerk verspricht Wall nicht nur höchste Design- und Materialqualität, sondern bietet den Städten entsprechend ihren spezifischen Anforderungen maßgeschneiderte Lösungen.\footnote{\label{foot:5}Vgl. ebd.} \\ Getreu dem Motto „Für Städte. Für Menschen.“ versteht sich Wall als Partner der Städte. Werbetreibenden soll eine innovative mediale Vielfalt zur Übermittlung ihrer Botschaft angeboten werden.\footnote{\label{foot:6}Wall GmbH: „Philosophie“, unter: http://www.wall.de/de/company/philosophy (abgerufen am 07.01.2018)} 

\subsection{Unternehmensbereiche}\label{sec:unternehmensbereiche}
Das Unternehmen gliedert sich in vier Unternehmensbereiche: Forschung \& Entwicklung, Produktion, Vermarktung und Reinigung \& Wartung. \\ Die Abteilung der Forschung und Entwicklung arbeitet daran, in Sachen Stadtmöblierung und Außenwerbung innovative, nachhaltige und kreative Lösungen zu schaffen. \\ In der Produktion hat die Präzision in der Fertigung oberste Priorität. Das Produktionswerk in Velten ist eines der modernsten Europas. Auf 10.000 qm arbeiten dort qualifizierte Mitarbeiter mit ausgefeilter Technik und hochwertigen Materialien. \\ Zum Bereich Reinigung und Wartung gehören ausschließlich eigens geschulte Mitarbeiter und ein ausgefeiltes Wartungs- und Reinigungskonzept. Dadurch kann gewährleistet werden, dass die Produkte jederzeit einwandfrei funktionieren und aussehen. Außerdem sind alle Produkte mit modernster Überwachungstechnik ausgestattet. Eine 24h-Hotline stellt sicher, dass Reparaturen umgehend ausgeführt werden können.\footnote{\label{foot:7}Wall GmbH: „Unternehmensbereiche“, unter: http://www.wall.de/de/company/unternehmensbereiche (abgerufen am 07.01.2018)} \\ Auf den Unternehmensbereich der Vermarktung gehe ich im nächsten Abschnitt ein. 

\subsection{WallDecaux}\label{sec:walldecaux}
Mein Praktikum ist im Unternehmensbereich der Vermarktung angesiedelt. In Deutschland wird das gemeinsame Portfolio von JCDecaux und der Wall GmbH unter der Vertriebsmarke „WallDecaux Premium Out of Home“ vermarktet. WallDecaux bündelt die Stärken der beiden familiengeführten Unternehmen, die beide in den vergangenen Jahrzehnten die Professionalisierung der Außenwerbung maßgeblich gestaltet haben.\footnote{\label{foot:8}Wall GmbH: „WallDecaux“, unter: http://www.wall.de/de/outdoor\_advertising/walldecaux (abgerufen am 07.01.2018)} \\ Durch die Zusammenführung der beiden Vertriebsbereiche ist WallDecaux zum Beispiel der größte Anbieter in Deutschland für das Format „City Light Poster“ und misst rund 49.000 CLP-Werbeflächen in mehr als 30 Städten.\footnote{\label{foot:9}Vgl. ebd. (abgerufen am 07.01.2018)} 

\section{Weg zur Praktikumsstelle}\label{sec:wegzurstelle}
Wenige Monate vor Beginn meines Praktikums suchte ich bereits nach einer neuen beruflichen Herausforderung, die zu meinen Interessensschwerpunkten und meiner Studienwahl passte. Dabei wurde ich auf die Wall GmbH aufmerksam und bewarb mich dort als Werkstudentin. Nach einem aufschlussreichen Bewerbungsgespräch, das mein Interesse weiter verstärkte, erhielt ich die Job-Zusage und arbeitete ab diesem Zeitpunkt als studentische Hilfskraft im digitalen Marketing der Wall GmbH. Sofern sich auch für den Zeitraum des Praktikums ein geeigneter Aufgabenbereich für mich finden ließe, standen mein Team und ich dieser Idee sehr aufgeschlossen gegenüber. \\ Als während meiner Einarbeitung als Werkstudentin klar wurde, dass es einen hohen Bedarf an Schnittstellentätigkeiten zwischen der IT-Abteilung und dem Marketing gibt, entschied ich mich, die Herausforderung anzunehmen und diese Aufgaben in einem dreimonatigen Vollzeitpraktikum auszuüben. Es bot sich dadurch für mich die Chance, einen Einblick in gleich zwei Unternehmensbereiche erhalten zu können. Die Interaktion zwischen IT-bezogenen Aufgabenbereichen und multimedialen (Kreativ-)Prozessen war für mich schon immer von großem Interesse und ausschlaggebend für meine berufliche Zukunft. In beiden Aufgabenfeldern sehe ich meine persönlichen Stärken, sodass dieses Praktikum mit Sicherheit Aufschluss über einen möglichen Berufswunsch geben und eine Rolle bei der späteren Berufswahl spielen würde. 

\newpage